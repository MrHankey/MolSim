              
% --------------------------------------------------------------
% This is all preamble stuff that you don't have to worry about.
% Head down to where it says "Start here"
% --------------------------------------------------------------

\documentclass[12pt]{article}

\usepackage[margin=1in]{geometry} 
\usepackage{amsmath,amsthm,amssymb}

\newcommand{\N}{\mathbb{N}}
\newcommand{\Z}{\mathbb{Z}}

\newenvironment{theorem}[2][Theorem]{\begin{trivlist}
\item[\hskip \labelsep {\bfseries #1}\hskip \labelsep {\bfseries #2.}]}{\end{trivlist}}
\newenvironment{lemma}[2][Lemma]{\begin{trivlist}
\item[\hskip \labelsep {\bfseries #1}\hskip \labelsep {\bfseries #2.}]}{\end{trivlist}}
\newenvironment{exercise}[2][Exercise]{\begin{trivlist}
\item[\hskip \labelsep {\bfseries #1}\hskip \labelsep {\bfseries #2.}]}{\end{trivlist}}
\newenvironment{reflection}[2][Reflection]{\begin{trivlist}
\item[\hskip \labelsep {\bfseries #1}\hskip \labelsep {\bfseries #2.}]}{\end{trivlist}}
\newenvironment{proposition}[2][Proposition]{\begin{trivlist}
\item[\hskip \labelsep {\bfseries #1}\hskip \labelsep {\bfseries #2.}]}{\end{trivlist}}
\newenvironment{corollary}[2][Corollary]{\begin{trivlist}
\item[\hskip \labelsep {\bfseries #1}\hskip \labelsep {\bfseries #2.}]}{\end{trivlist}}

\begin{document}

% --------------------------------------------------------------
%                         Start here
% --------------------------------------------------------------

%\renewcommand{\qedsymbol}{\filledbox}

\title{Excercise 1}%replace X with the appropriate number
\author{Raphael Leiteritz\\ %replace with your name
Molecular Simulation} %if necessary, replace with your course title

\maketitle

\section{Task 1}

Given a square with length $l$ and a circle lying within that circle with radius $r = \frac{l}{2}$ we can define the surface ratio $r_s$ as
\begin{equation}
	r_s = \frac{\pi r^2}{(2r)^2}
\end{equation}
re-arraging terms yields in an expression for
\begin{equation}
\qquad	\pi = 4r_s
\end{equation}

The surface ration $r_s$ can easily be approximated using Monte Carlo methods. First you draw $N$ uniformly distributed random samples within the square. Then the surface ratio is approximated by the quotient of number of samples that lie within the circle $N_{in}$ and the total number of samples generated $N$.

\begin{equation}
	r_s \approx \frac{N_{in}}{N}
\end{equation}

Having the diameter of the circle not readily available but rather implicitly defined by the length ratio $rat = \frac{l}{d}$ changes our surface ration equation to

\begin{align*}
	r_s &= \frac{\pi r^2}{l^2} rat^2 \\
	\Leftrightarrow \qquad \pi &= 4r_s*
\end{align*}

\end{document}