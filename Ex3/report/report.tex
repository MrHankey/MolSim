
\documentclass[a4paper,12pt]{article}
\usepackage{fancyhdr}
\usepackage{fancyheadings}
\usepackage[utf8]{inputenc}
%\usepackage[latin1]{inputenc}
\usepackage[active]{srcltx}
\usepackage{algorithm}
\usepackage[noend]{algorithmic}
\usepackage{amsmath}
\usepackage{amssymb}
\usepackage{amsthm}
\usepackage{bbm}
\usepackage{enumerate}
\usepackage{graphicx}
\usepackage{ifthen}
\usepackage{listings}
\usepackage{struktex}
\usepackage{hyperref}
\usepackage{varioref}

%%%%%%%%%%%%%%%%%%%%%%%%%%%%%%%%%%%%%%%%%%%%%%%%%%%%%%
%%%%%%%%%%%%%% EDIT THIS PART %%%%%%%%%%%%%%%%%%%%%%%%
%%%%%%%%%%%%%%%%%%%%%%%%%%%%%%%%%%%%%%%%%%%%%%%%%%%%%%
\newcommand{\Fach}{Molekularsimulation}
\newcommand{\Name}{Raphael Leiteritz}
\newcommand{\Seminargruppe}{}
\newcommand{\Matrikelnummer}{2787596}
\newcommand{\Semester}{WS 16/17}
\newcommand{\Uebungsblatt}{3} %  <-- UPDATE ME
%%%%%%%%%%%%%%%%%%%%%%%%%%%%%%%%%%%%%%%%%%%%%%%%%%%%%%
%%%%%%%%%%%%%%%%%%%%%%%%%%%%%%%%%%%%%%%%%%%%%%%%%%%%%%

\setlength{\parindent}{0em}
\topmargin -1.0cm
\oddsidemargin 0cm
\evensidemargin 0cm
\setlength{\textheight}{9.2in}
\setlength{\textwidth}{6.0in}

%%%%%%%%%%%%%%%
%% Aufgaben-COMMAND
\newcommand{\Aufgabe}[1]{
	{
		\vspace*{0.5cm}
		\textsf{\textbf{Aufgabe #1}}
		\vspace*{0.2cm}
		
	}
}
%%%%%%%%%%%%%%
%\hypersetup{
%	pdftitle={\Fach{}: Übungsblatt \Uebungsblatt{}},
%	pdfauthor={\Name},
%	pdfborder={0 0 0}
%}

\lstset{ %
	language=java,
	basicstyle=\footnotesize\tt,
	showtabs=false,
	tabsize=2,
	captionpos=b,
	breaklines=true,
	extendedchars=true,
	showstringspaces=false,
	flexiblecolumns=true,
}

\title{Übungsblatt \Uebungsblatt{}}
\author{\Name{}}

\begin{document}
	\thispagestyle{fancy}
	\lhead{\sf \large \Fach{} \\ \small \Name{} - \Matrikelnummer{}}
	\rhead{\sf \Semester{}}
	\vspace*{0.2cm}
	\begin{center}
		\LARGE \sf \textbf{Übungsblatt \Uebungsblatt{}}
	\end{center}
	\vspace*{0.2cm}
	
	%%%%%%%%%%%%%%%%%%%%%%%%%%%%%%%%%%%%%%%%%%%%%%%%%%%%%%
	%% Insert your solutions here %%%%%%%%%%%%%%%%%%%%%%%%
	%%%%%%%%%%%%%%%%%%%%%%%%%%%%%%%%%%%%%%%%%%%%%%%%%%%%%%
	
	\Aufgabe{1}
	\begin{enumerate}[a)]
		\item Looking at \vref{fig:chem_pot} the vapor liquid coexistence point lies at a densities $rho = 0.4$ and $\rho_2 = 0.007$. The pressure is $P \approx 0.005313$ and chemical potential $\mu \approx -4.056$.
		
		Calculating the chemical potential at high densities is more difficult more difficult than at lower ones because at high densities the term
		\begin{equation}
			ln\left(\frac{ \left\langle exp[-\beta \Delta U^{+}] \right\rangle }{\rho}\right)
		\end{equation}
		
		is influencing the overall result more and thus it is more important to estimate the inner term better by investing more computation time.
		
		0.0071
		0.7424
		
		\item Performing a simulation in the Gibbs ensemble using the provided code yielded a vapor-liquid coexistence point at around $P = 0.0103$ and $\mu \approx -4.1440$.
		
		The coexistence densities are $\rho_1 = 0.0071$ and $\rho_2 = 0.7424$
		
		I think it is safe to say that while the chemical potential of both methods almost agrees, the second calculated density certainly does not. This may have multiple causes. First of all the Gibbs simulation did not fully converge. The chemical potentials in both boxes were still $0.1-0.2$ apart from each other. Simply simulating more cycles and more particles may solve this problem.
		
		Other than that, the first method may be  
		
	\end{enumerate}
	
	\begin{figure}[ht]
		\centering
		\includegraphics[width=\textwidth]{chem_pot.eps}
		\caption{Chemical potential vs. pressure calculated for different densities.}
		\label{fig:chem_pot}
	\end{figure}
	
	%%%%%%%%%%%%%%%%%%%%%%%%%%%%%%%%%%%%%%%%%%%%%%%%%%%%%%
	%%%%%%%%%%%%%%%%%%%%%%%%%%%%%%%%%%%%%%%%%%%%%%%%%%%%%%
\end{document}



              
%% --------------------------------------------------------------
%% This is all preamble stuff that you don't have to worry about.
%% Head down to where it says "Start here"
%% --------------------------------------------------------------
%
%\documentclass[12pt]{article}
%
%\usepackage[margin=1in]{geometry} 
%\usepackage{amsmath,amsthm,amssymb}
%\usepackage{graphicx}
%\usepackage{varioref}
%
%\newcommand{\N}{\mathbb{N}}
%\newcommand{\Z}{\mathbb{Z}}
%
%\newenvironment{theorem}[2][Theorem]{\begin{trivlist}
%\item[\hskip \labelsep {\bfseries #1}\hskip \labelsep {\bfseries #2.}]}{\end{trivlist}}
%\newenvironment{lemma}[2][Lemma]{\begin{trivlist}
%\item[\hskip \labelsep {\bfseries #1}\hskip \labelsep {\bfseries #2.}]}{\end{trivlist}}
%\newenvironment{exercise}[2][Exercise]{\begin{trivlist}
%\item[\hskip \labelsep {\bfseries #1}\hskip \labelsep {\bfseries #2.}]}{\end{trivlist}}
%\newenvironment{reflection}[2][Reflection]{\begin{trivlist}
%\item[\hskip \labelsep {\bfseries #1}\hskip \labelsep {\bfseries #2.}]}{\end{trivlist}}
%\newenvironment{proposition}[2][Proposition]{\begin{trivlist}
%\item[\hskip \labelsep {\bfseries #1}\hskip \labelsep {\bfseries #2.}]}{\end{trivlist}}
%\newenvironment{corollary}[2][Corollary]{\begin{trivlist}
%\item[\hskip \labelsep {\bfseries #1}\hskip \labelsep {\bfseries #2.}]}{\end{trivlist}}
%
%\begin{document}
%
%% --------------------------------------------------------------
%%                         Start here
%% --------------------------------------------------------------
%
%%\renewcommand{\qedsymbol}{\filledbox}
%
%\title{Excercise 3}%replace X with the appropriate number
%\author{Raphael Leiteritz\\ %replace with your name
%Molecular Simulation} %if necessary, replace with your course title
%
%\maketitle
%
%\section{Task 1}
%Looking at \vref{fig:chem_pot} the vapor liquid coexistence point lies at a density $\rho^* \approx 0.033$.
%
%
%Calculating the chemical potential at high densities is more difficult more difficult than at lower ones because at high densities the term
%\begin{equation}
%ln\left(\frac{ \left\langle exp[-\beta \Delta U^{+}] \right\rangle }{\rho}\right)
%\end{equation}
%
%is influencing the overall result more and thus it is more important to estimate the inner term better by investing more computation time.
%
%\begin{figure}[ht]
%    \label{fig:chem_pot}
%	\centering
%  \includegraphics[width=\textwidth]{chem_pot.eps}
%	\caption{Chemical potential vs. pressure calculated for different densities.}
%	\label{fig2}
%\end{figure}
%
%\section{Task 2}
%
%Running a Gibbs
%
%\end{document}